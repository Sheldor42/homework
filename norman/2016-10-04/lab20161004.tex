\documentclass[12pt]{article}
\usepackage{fullpage,amsmath,amsfonts}

\begin{document}

\title{Stochastic Methods + Lab}
\author{Session 10}
\date{October 4, 2016}
\maketitle

\begin{enumerate}


\item Rewrite your code so that it stores the option value at each
node of the tree.  Then visualize the tree using \texttt{imshow}.
Think about an appropriate color map, how to mask the missing values
(Hint: use Numpy's masked arrays), and how to best map the computed
values to pixel coordinates.

\item The price of a European Call option with current stock price
$S$, strike price $K$, annualized volatility $\sigma$, annual
risk-free interest rate $r$, and maturity time $T$ can be computed
explicitly from the Black--Scholes formula
\begin{equation*}
  C = S \, \Phi(x) - K \, \mathrm{e}^{-rT} \, \Phi(x-\sigma \sqrt T) \,,
\end{equation*}
where 
\begin{equation*}
  x = \frac{\ln(S/K) + (r + \sigma^2/2) \, T}{\sigma \sqrt T}
\end{equation*}
and $\Phi$ denotes the cumulative distribution function of the
standard normal distribution with mean zero and variance one.

Compare your call option prices from the binomial tree model against
those computed by the Black--Scholes formula.  Does the error scale
like a power of $N$?  (Plot the logarithm of the error vs.\ $N$.  Do
you obtain a straight line?)

\item Show that an American call option should be exercised at
expiration for maximal profit.

\item Modify your binomial tree algorithm to price an American put
option (the holder may exercise the option at any time before
expiration).  Is the price of an American put higher or lower than
that that of a European put with otherwise identical parameters?

\end{enumerate}


\end{document}

