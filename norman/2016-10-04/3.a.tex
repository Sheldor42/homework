\documentclass{article}
\usepackage[utf8]{inputenc}
\usepackage{listings}
\usepackage{amsmath}
\usepackage{amsfonts}
\usepackage{scrextend}
\usepackage{float}
\usepackage{graphicx}
\newsavebox{\mybox}
%\addtolength{\hoffset}{-1.5cm}
\addtolength{\voffset}{-2cm}
%\addtolength{\topmargin}{-2cm}
%\addtolength{\textwidth}{2.5cm}
%\addtolength{\textheight}{7cm}
\title{Stochastics Methods + Lab - homework\\[0.5cm]
\author{Norman Heil}
\large Assignment Sheet form 2016-10-04}
\date{Oktober 4th. 2016}
\maketitle
\begin{document}
\section*{3. Show that an American call option should be exercised at expiration for maximal profit.}
Looking at the binomial pricing model, which for some $N:\forall n>N\element \in \mathbb{N}$ the price $x_n$ converges uniformly to the true price calculated using the Black–Scholes formula, one can easily see that for a call option after each time step which is not the final time step, there exist a possible outcome with an higher yield.\\

Going away from the binomial tree model, one might also argue that for maximum profit assuming a monotonically increasing yield over time, assumed by the model of a risk neutral world and the given interest rate $r$, as long as $r$ lies above the rate of inflation, exercising a call option later would be more preferable than exercising the option earlier. This is, as one can see that the sequence of the true yield $X'$ over $n'$ options $X_{n'}' \longrightarrow X_{n'}''$ where $X''$ is the theoretical yield over $n'$ options calculated using the risk free interest rate $r$.


\end{document}
