\documentclass[12pt]{article}
\usepackage{fullpage,amsmath,amsfonts}

\begin{document}

\title{Stochastic Methods + Lab}
\author{Session 2}
\date{September 6, 2016}
\maketitle

\noindent\emph{Note:} The work is to be submitted up to one week after
the session via \texttt{git}.  For this task sheet, it is sufficient
to submit the code so long as it runs and produces the
requested output.

\begin{enumerate}

\item An investment is guaranteeing a cash flow $C_1, \dots, C_N$ at
the end of each period.  The period interest rate is $y$.  Write
python functions to compute the present value of the investment in
three different ways:
\begin{enumerate}
\item Using Numpy arrays;
\item Using the \texttt{polyval} function;
\item By writing an explicit Python loop.
\end{enumerate}
Compare the run-time of the three implementations on the following
test case:
\begin{verbatim}
  C = 100.0 * arange(3,2003)
  y = 0.05
\end{verbatim}

% \item (From Ross, p.\ 45.) An individual who plans to retire in 20
% years has decided to put an amount $A$ in the bank at the beginning of
% each of the next 240 months, after which she will withdraw EUR 1000
% at the beginning of each of the following 360 months.  Assuming a
% nominal yearly interest rate of 6\% compounded monthly, how large does
% $A$ need to be?

\item Write a Python program which prints out an amortization schedule
for a mortgage. 

The program should take as input the nominal yearly interest
rate $r$, the amount of the loan $P$, the number of compounding
periods per annum $m$, and the term of the mortgage $n$ in years.
Assume that the mortgage is fully redeemed at the end of the term.

The program should compute the monthly payment, the effective annual
interest rate, and a detailed payment schedule listing, for each
month showing the interest and principal parts of the payment and the
remaining principal.

Run your program with $P=250\,000$, $r=0.08$,
$m=12$, and $n=15$.

% \item Find a closed form formula for the remaining principal right
% after the $k$th mortgage payment.

\item An investment sold at price $P$ is guaranteeing a cash flow
$C_1, \dots, C_N$ at the end of each year.  Write a program to compute
its IRR (internal rate of return).

Run your program on the following test case:
\begin{verbatim}
  N = 20
  C = 100.0 * arange(3,N+3)
  P = 20000.0
\end{verbatim}

\end{enumerate}


\end{document}

