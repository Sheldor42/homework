\documentclass[12pt]{article}
\usepackage{fullpage,amsmath,amsfonts}

\def\e{\mathrm e}
\def\d{\mathrm d}

\begin{document}

\title{Stochastic Methods + Lab}
\author{Session 14}
\date{October 18, 2016}
\maketitle

\begin{enumerate}


\item On the previous task sheet, you computed an ensemble of
geometric Brownian paths
\[
  S(t) = \exp((\mu - \tfrac12 \, \sigma^2)\, t + \sigma \, W(t))
\]
with $\mu=0.05$ and $\sigma=0.3$ and plot mean and standard deviation as
a function of time on the interval $[0,1]$.

Now add, into the same coordinate system, mean and standard deviation
of the stock price paths which underlie the binomial tree model with
$N=500$ time steps calibrated with the same set of parameters $r=\mu$
and annualized volatility $\sigma$.  What do you see?

\item Use the paths so obtained in a Monte--Carlo valuation of a
European call option with $K=0.9$, time of maturity $T=1.0$ and risk
free rate $r=\mu$.  Compare your result against the Black--Scholes
price by plotting the deviation from the Black--Scholes price against
the number of samples in a doubly logarithmic plot.

What is the order of the Monte--Carlo method as a function of the
number of samples?

\item Look up stock option quotes for European call options on the
stock of a major corporation.  Plot the implied volatility vs.\ the
strike price, while the time to maturity is fixed.  (The applicable
interest rate is the spot rate for zero coupon bonds of the same
maturity.)

\item Approximate the It\=o integral
\[
  I = \int_0^T X(t-) \, \d W(t)
    = \lim_{N\to\infty} \sum_{i=0}^{N-1} X(t_i) \, (W(t_{i+1}) -
    W(t_i))
\]
and the Stratonovich integral
\[
  W = \int_0^T X(t) \, \d W(t)
    = \lim_{N\to\infty} 
      \sum_{i=0}^{N-1} X\Bigl(\frac{t_{i+1}+t_i}2 \Bigr) 
      \, (W(t_{i+1}) - W(t_i)) \,,
\]
where $W(t)$ denotes standard Brownian motion, $t_i = i \, \Delta t$
with $\Delta t = T/N$, and we choose, as an example, $X(t)=W(t)$.   Do
they converge to the same value as $N \to \infty$?

\end{enumerate}



\end{document}

