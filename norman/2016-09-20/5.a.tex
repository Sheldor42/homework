\documentclass{article}
\usepackage[utf8]{inputenc}
\usepackage{amsmath}

\title{Stochastics methods lab - homework\\[0.5cm]
\large Problem 2016-09-20 number 5}

\author{Norman Heil}
\date{September 28th. 2016}

\begin{document}

\maketitle
\section{Problem}
Visit the web site of the European Central Bank (ECB) and look for their yield curve data. What data is contained in the files?
\subsection{Abstract}
Looking at the .csv file provided by the ECB one can find out that the data in general regards yield curves of government issued bonds.\\
The date of the file is provided (2016-09-27), then the data is in general ordered by the categories of AAA-rated bonds, what refers to the issuers credit rating, and all central governments bonds in the euro area.\\
Next the bonds yield is calculated using their spot rates, par yield rates and instantaneous forward rates. They are given for the respective bonds, which are identified by their respective series keys.\\
The description for each bond tells, that the all of the bonds are issued by governments in the euro zone, and if they have a AAA rating, that they use the Svensson model are continuously compounded yield error minimisation has been applied and the residual maturity for the corresponding bond. Finally the description also says that the bonds are in Euro and data is provided by the ECB.


\end{document}
