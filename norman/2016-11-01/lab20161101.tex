\documentclass[12pt]{article}
\usepackage{fullpage,amsmath,amsfonts,graphicx}

\begin{document}

\title{Stochastic Methods + Lab}
\author{Session 17}
\date{November 1, 2016}
\maketitle

\def\e{\mathrm e}
\def\d{\mathrm d}


Let $X=X(t)$ be an It\=o process, i.e., a solution of the
stochastic differential equation
\[
  \d X = f(X,t) \, \d t + g(X,t) \, \d W \,,
\]
interpreted in the sense of the It\=o stochastic integral.  Let
$F(X,t)$ be twice continuously differentiable.
Then the stochastic chain rule, also known as the It\=o formula, reads
\[
  \d F(X,t)
  = \biggl(
      \frac{\partial F(X,t)}{\partial t}
      + f(X,t) \, \frac{\partial F(X,t)}{\partial X}
      + \frac12 \, g(X,t)^2 \, \frac{\partial^2 F(X,t)}{\partial X^2}
    \biggr) \, \d t
  + g(X,t) \, \frac{\partial F(X,t)}{\partial X} \, \d W \,.
\]

Verify the It\=o formula numerically for the example when $X(t)$ is
geometric Brownian motion with $\mu = 0.2$ and $\sigma=2.0$, and where
\[
  F(X,t) = (1+t) \, \sqrt X \,.
\]

\emph{Hint:} You have to compare direct evaluation of this expression
with a numerical solution of the stochastic differential equation
which you obtain from the It\=o formula.

\end{document}

