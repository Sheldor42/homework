\documentclass[12pt]{article}
\usepackage{fullpage,amsmath,amsfonts}

\begin{document}

\title{Stochastic Methods + Lab}
\author{Session 4}
\date{September 13, 2016}
\maketitle

\noindent\emph{Note:} The computational work is to be submitted up to
one week after the session via \texttt{git}.  For this task sheet, it
is sufficient to submit the code so long as it runs and produces the
requested output.  Theoretical questions may be submitted handwritten
on paper.

\begin{enumerate}


\item Use \texttt{timeit} to compare the efficiency of Newton's
method, the secant method, and Brent's method for computing the IRR of
the test case from Lab Session 2. Repeat for $N=200$ and
$P=1\,500\,000$.

\item The \emph{yield to maturity} of a level coupon bond is the IRR
of its cash flow.  Compute the yield to maturity of a 10-year level
coupon bond sold at 75\% of par with a coupon rate of 10\% paid
semiannually.

\item Plot the price vs.\ time to maturity for level coupon bonds with
annual coupon rates of 2\%, 6\%, and 12\% paid semiannually.  Assume a
yield of 6\% and a par value of EUR 1\,000.


\item (From Ross, p.\ 45.) An individual who plans to retire in 20
years has decided to put an amount $A$ in the bank at the beginning of
each of the next 240 months, after which she will withdraw EUR 1000
at the beginning of each of the following 360 months.  Assuming a
nominal yearly interest rate of 6\% compounded monthly, how large does
$A$ need to be?

\item Suppose the coupon rate for a level coupon bond is the same as
the market rate.  Show that this bond will be sold at par.


\end{enumerate}


\end{document}

